\chapter*{Introduction}
\addcontentsline{toc}{chapter}{Introduction}
\markboth{Introduction}{Introduction}
\label{chap:introduction}
%\minitoc

Selon une étude de l'Alliance pour les Chiffres de la Presse et des Médias (ACPM) \cite{oneGlobal}. En 2016, les Français ont lu 5,6 titres différents (quotidiens et magazines confondus) chaque mois en moyenne. Ce sont les femmes qui lisent plus de titres que la moyenne tout comme les « ultraconnectés». Les versions papiers représentent 51 \% des lectures contre 49 \% pour les versions numériques dont le pourcentage augmente d'année en année sur la version papier. Cette tendance est particulièrement marquée chez les personnes âgées de 15 à 24 ans privilégient le mobile et dont 55 \% d'entre eux lisent au moins une marque de presse sur mobile. Les marques de presse ont compris qu'il fallait toucher le maximum de personnes sur les applications mobiles dont la tranche des 15-24 ans utilisent le plus, c’est-à-dire les réseaux sociaux. Sur les réseaux sociaux, c'est sur Twitter que les marques de presse sont les plus suivies avec 55 \% des personnes interrogées utilisant régulièrement Twitter suivent une marque de presse suivit par Google+ et Facebook. Si la majorité de ces internautes suivent une marque de presse, c'est pour suivre l'actualité en temps réel à 72 \%, partager les articles avec leurs contacts et trouver des informations qu'ils ne trouvent pas ailleurs.  

Dans le sous-domaine de la recommandation d’articles d’actualités, il existe à ce jour très peu de systèmes permettant aux différentes équipes de recherche de tester leurs algorithmes en temps réel sur des utilisateurs interagissant directement avec le système de recommandation. L'idée de l'application Renewal est née de ce constat mais ne veux pas se cantonner au monde de la recherche. La finalité est d'avoir une application utilisée par un grand nombre d'utilisateur permettant la récolte et l'analyse fine du comportement de chaque  utilisateur. Le but est double, pour les utilisateurs recevoir un feed d'actualité pertinent avec ses gouts et pour les équipes de recherches déterminer l'algorithme de recommandation le plus pertinent.  

Dans le cadre de ma première année de Master MIAGE (Méthodes Informatiques Appliquées à la Gestion des Entreprises à l'Université de Paris Nanterre, j'ai souhaité réaliser un stage me permettant de me former sur le développement d'application mobile en complément de ce qui m'as été dispensé durant ma formation. J'ai effectué ce stage au sein du Laboratoire de Recherche en Informatique du 7 avril au 31 juillet 2018. 

%%% Local Variables: 
%%% mode: latex
%%% TeX-master: "lri-report"
%%% End: 
