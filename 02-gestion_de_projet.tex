\chapter{Plannification et gestion de projet}
\label{sec:Plannification et gestion de projet}

\section{Contexte}

La réalisation de l'application s'est effectuée de façon agile. L'agilité représente un ensemble de méthodes, de process, fortement orienté collaboratif avec une certaine transversalité et autonomie. Les grands principes de l'agilité ont été notamment exposés dans le Manifeste Agile. Ici le "Product Owner", mon tuteur J. HAY représente les intérêts du client et à ce titre, il a l’autorité pour définir les fonctionnalités du produit final. Il est responsable du "Backlog", une liste des tâches et des spécificités du produit (le cahier des charges). La mise en place d'un backolog priorisé s'est effectué avec l'utilisation de l'outil Trello. Le logiciel Trello est disponible sur de multiple plateforme et permet donc d'être notifié simplement à chaque changement ou avancée du projet. Ensuite, mon travail s'est articulé autour de différents Sprint. Dans la méthode agile de gestion de projet, l'utilisation des sprints est utilisé intervalles de temps pendant lesquels une équipe va compléter un certain nombre de tâches du backlog.Chaque  jours, a eu lieu un "Daily Meating", il s'agit d'une réunion quotidienne où chacun fait part aux autres de son avancement sur le projet. Ce daily meating était soit effectué directement sur notre lieu de travail soit via l’application Discord. Chaque sprint terminé donne lieu à une rétrospective, ce moment permet de réunir toute l’équipe donc ici J. HAY, A. JODAR (un autre stagiaire travaillant sur la récupération des actualités lié à mon application) et moi-même afin de partager les retours d’expérience et discuter des améliorations possibles du prochain sprint.

\section{L’agilité}
Les pratiques d’ingénierie agiles reposent sur des compétences, des pratiques et des concepts que j’ai appliqué lors de mon stage durant la réalisation de l’application mobile :

\begin{itemize}
    \item Le refactoring : consiste à modifier un code source de façon à en améliorer la structure du code sans que cela ne modifie les fonctionnalités. Le but premier du refactoring est d’avoir un code réactif aux changements et plus facilement maintenable. 
    
    \item  Le développement incrémental : consiste à réaliser successivement des éléments fonctionnels utilisables. Cette pratique permet de gagner un temps important comparé aux méthodes de gestion de projet dites en cascades qui ne livrent qu’à la fin d’un long processus de production. Ici cette méthode de développement permet de livrer à la fin de chaque sprint une application mobile fonctionnelle avec des fonctionnalités utilisables. 
    
    \item Les livraisons fréquentes : une équipe agile met fréquemment son produit entre les mains des utilisateurs finaux, aptes à l’évaluer et à formuler des critiques et des appréciations. Cela crée une dynamique entre l’équipe et ses clients et instaure une relation de confiance durable. Lors de mon stage, le product owner représenté par J. HAY à pu tester régulièrement l’application au cours de son développement et plus particulièrement lors de la livraison de l’application à la fin de chaque sprint.
\end{itemize}


\section{Outil de communication}
 
Discord est un logiciel conçu initialement pour la communauté des joueurs en ligne mais ce logiciel est parfaitement adapté au monde de l'entreprise et y trouve aisément sa place. Extrêmement simple d’utilisation, il fonctionne sur un principe de serveurs, dans un contexte d'entreprise avec une multitude de projet cela permet de classer facilement les conversations par projet ou groupe de travail. La communication au sein d'un serveur est découpée dans un ensemble de salons vocaux ou textuels appelé « channels » . Discord offre la possibilité de créer des « channels » adaptés aux besoins de notre équipe, il est possible de créer un channel « général » ou tous pourront échanger, un channel « React-Native » réservé à mon travail et l'avancement régulier lors que mon tuteur était absent ou en réunion. 

%%% Local Variables: 
%%% mode: latex
%%% TeX-master: "lri-report"
%%% End: 